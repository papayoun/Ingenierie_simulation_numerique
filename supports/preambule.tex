Ces notes ont pour but de présenter certaines méthodes de simulations dans un but d'applications en statistiques.
Nous présenterons d'abord les méthodes pour la simulation de variables aléatoires à partir d'un ordinateur. 
Nous parlerons ensuite de méthode de Monte Carlo pour l'intégration. 
Plus précisément, nous verrons comment les méthodes de simulation permettent de faire du calcul approché d'intégrales. 
Enfin, nous parlerons de l'utilisation des outils de simulation de loi pour l'inférence statistique, et spécifiquement dans le cadre Bayésien.

\vspace{\baselineskip}

Ces notes n'ont rien d'original, elles ne font que reprendre, de manière souvent moins exhaustive, des cours existants.

J'ai en effet été puiser dans les cours de différents confrères, en premier lieu dans le cours d'Arnaud Guyader (Sorbonne Université) \citep{guyader2018methodes}, mais également de Bernard Delyon (Université Rennes I)  \citep{delyon2017simulation}, Sylvain Rubenthaler (Université Nice Sophia Antipolis) \citep{rubenthaler2018methodes}, Julien Stoehr (Paris-Dauphine) et Sylvain le Corff (Télécom Paris). Je remercie ces auteurs pour le libre accès à leurs notes, qui facilite grandement la diffusion du savoir. Les notes que j'écris sont évidemment libre d'accès et de diffusion en ce sens.

De même la série d'exercices proposées en TD est souvent un échantillon des exercices déjà proposés dans ces références. 